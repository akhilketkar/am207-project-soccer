\title{Inefficiencies in the Soccer Betting Market}
\author{
        Owen Prunskis \\
            \and
        Akhil Ketkar\\
}
\date{\today}

\documentclass[12pt]{article}

\begin{document}
\maketitle

% \begin{abstract}
% This is the paper's abstract \ldots
% \end{abstract}

\section{Introduction}
Soccer is the most popular sport in the world, particularly in Europe. The 
popularity of the sport combined with the legal status of gambling has led
to the formation of a very deep and active betting market for soccer. A
number of different bookmakers allow people to bet large sums of money 
on various aspects of the game. One can obviously bet on the outcomes
of games but also other scenarios like the score at half-time, number of 
penalties, etc. The objective our project is to find actionable inefficiencies in the 
soccer betting market. 

% \paragraph{Outline}
%The remainder of this article is organized as follows.
%Section~\ref{previous work} gives account of previous work.
%Our new and exciting results are described in Section~\ref{results}.
%Finally, Section~\ref{conclusions} gives the conclusions.

\section{Previous work}\label{previous work}
A fair amount of work has been done in this area. But it pales in comparison
to the amount of research done on American sports, especially baseball.
The most important work for the purposes of our project is the Dixon-Coles
1997 paper \cite{dixoncoles}. Additional models and expansions are included in the
bibliography as well. \cite{constantinou-1} \cite{constantinou-2} \cite{constantinou-3}

\section{Methodology}\label{methodology}
Dixon and Coles propose a bayesian model that uses a bi-variate Poisson 
distribution for number of goals scored by each team. We will use this model 
as the basis of our inquiry and improve upon it as needed. In particular we might 
look at  differences in scoring in the first half of a game compared to the second half, 
the effect of yellow and red cards, the effect of multiple games within the same
week, of injuries to certain players, of coaches etc.

\par 
We will train the model on available data and then use 
MCMC methods to sample from the distributions of the model parameters. Finally
we will test the model against bookmakers' odds to see if they can be systematically
beaten. 

\section{Data}\label{data}
The data we will use for this project will primary come from the various divisions of
English Football focusing on the Premier League which is the top tier of competition.
Although one might suspect the betting market in smaller teams to be less efficient. 
The data comes from the website Football Data \cite{football-data} which has data for
all divisional games for multiple years along with the bookmakers odds for each game.
Additionally, we will leverage the existing EPL ipython notebook, which implements the basic
model and will allow for more flexibility in improving upon the existing work in the field.
\cite{EPL-notebook}

\section{Expansions and Improvements}\label{expansions and improvements}
Existing models typically generate a clustering effect, which tends to normalize 
outcomes. We look to incorporate additional parameters to address this shortcoming. We 
consider the following additions:

\begin{enumerate}
\item {\em Game Importance}, as defined by a threshold point distance away from 
winning the championship or the lower bound for relegation.

\item {\em Midweek Games}, to capture the effects of player fatigue.

\item {\em Weather}, binary as 'clear' or 'rain', which has a marked effect in other sports 
litrerature (i.e. baseball)

\item {\em Roster Cost}, amount spent on players and its correlation to improved results 
(QPR being an obvious outlier)

\item {\em Player Strength}, allows for more granularity in assessing player performance
\end{enumerate}

\par
We also hope to assess the validity of the model by exploring intergame data. We 
hypothesize that modeling goal scoring as poisson is not supported by data, and 
we hope to fit a more appropriate distribution to the model (such as one distributed 
over a more reasonable support). Lastly, we hope to investigate the accuracy of the validity
of the FIFA league coefficients.



\begin{thebibliography}{1}

	\bibitem{dixoncoles} Mark J. Dixon, Stuart G. Coles {\em Modeling Association
	Football Scores and Inefficiencies in the Football Betting Market} Applied Statistics,
	Volume 46, Issue 2 (1997) 265-280
	
	\bibitem{football-data} http://www.football-data.co.uk/englandm.php
	
	\bibitem{EPL-notebook} EPL iPython Notebook
	
	\bibitem{constantinou-1} Constantinou, A., N. E. Fenton and M. Neil (2013) {\em Profiting
	from an Inefficient Association Football Gambling Market: Prediction, Risk and 
	Uncertainty Using Bayesian Networks}.
	
	\bibitem{constantinou-2} Constantinou, A., N. E. Fenton and M. Neil (2012). {\em pi-football:
	A Bayesian network model for forecasting Association Football match outcomes}.
	
	\bibitem{constantinou-3} Constantinou, A. , Fenton, N.E., {\em Solving the problem of inadequate
	scoring rules for assessing probabilistic football forecasting models}, Journal of Quantitative 
	Analysis in Sports, Vol. 8, Article 1, 2012.
	
\end{thebibliography}

\end{document}
